\chapter{Demonstration: Global Illumination}

\section{Results obtained}
\label{ResultsObtained}

\par
The developed application was tested in five different devices:

\begin{itemize}
	\item Samsung Galaxy Fresh Duos GT-S7392
	\item Raspberry Pi 2 Model B
	\item MINIX NEO X8-H PLUS (k200)
	\item Android Virtual Device (AVD) in an HP Z440 desktop
	\item Android Virtual Device (AVD) in a Clevo W230SS laptop
\end{itemize}

\begin{table}[H]
	\small
	\centering
	\caption{Devices specifications.}
	\label{specs}
	\hspace*{-3cm}
	\begin{tabular}{|l|l|l|l|l|}
		\hline
		Device&CPU&Cache(L1/L2/L3)&GPU&RAM\\ \hline
		Samsung Galaxy&1xARM Cortex A9 @1GHz&64KB/Unknown&1xBroadcom VideoCore IV&512MB\\ \hline
		Raspberry Pi 2 Model B&4xARM Cortex A7 @900MHz&64KB/1MB&1xBroadcom VideoCore IV&1GB\\ \hline
		MINIX NEO X8 PLUS&4xARM Cortex A9 @2.0GHz&64KB/Unknown&4xMali-450 MP&2GB\\ \hline
		HP Z440 desktop&4xIntel Xeon E5-1620 v4 @3.5GHz&64KB/256KB/10MB&1xNvidia Quadro P2000&16GB\\ \hline
		Clevo W230SS laptop&4xIntel® Core™ i7-4710MQ @2.5GHz&64KB/256KB/6MB&1xNvidia 860M&16GB \\ \hline
	\end{tabular}
\end{table}

\par
As you can see in the table \ref{specs}, the application has been tested on a variety of devices.
Unfortunately, it was only possible to test it on one mobile device, the Samsung Galaxy Fresh Duos GT-S7392.
This device is a low-end smartphone with a low-end single core CPU.

\par
It was also possible to test it on two different computers that are portable and even smaller than a common laptop.
The Raspberry Pi 2 Model B and the MINIX NEO X8-H PLUS which are devices with the Android operating system installed.

\par
Last, but not least, it was also tested in two Android Virtual Devices, one desktop and one laptop.
The desktop is a HP Z440 computer, which is a workstation, and the laptop is a Clevo W230SS, which is a middle end laptop.

\begin{figure}[H]
	\centering
	\caption{Illustration of conference scene, rendered with shader Whitted in MobileRT.}
	\label{scene_conference}
	\includegraphics[keepaspectratio,scale=0.5]{Scene_Conference.png}
\end{figure}

\par
The scene used for testing was the conference scene, as illustrated in figure \ref{scene_conference}.
This scene consists of 331179 triangles and has two area lights in form of triangles.

\subsection{Raspberry Pi 2 Model B}

\begin{tikzpicture}
\begin{axis}[
legend style={at={(1.05,1.00)},anchor=north west},
axis lines = left,
xlabel = \#threads,
ylabel = speedup,
xtick={0,1,2,3,4},
ytick={0,1,...,10},
]
\addplot [
color=blue,
mark=*,
dashed,
] plot coordinates {
	(0,0.0)
	(1,1.0)
	(2,1.0)
	(3,1.0)
	(4,1.0)
};
\addlegendentry{Without Accelerator Structure}
\addplot [
color=black,
mark=*,
dashed,
] plot coordinates {
	(0,0.0)
	(1,1.0)
	(2,1.0)
	(3,1.0)
	(4,1.0)
};
\addlegendentry{Regular Grid}
\addplot [
color=red,
mark=*,
dashed,
] plot coordinates {
	(0,0.0)
	(1,1.0)
	(2,1.0)
	(3,1.0)
	(4,1.0)
};
\addlegendentry{Octree}
\label{graph:NoShadows}
\end{axis}
\end{tikzpicture}


\subsection{MINIX NEO X8-H PLUS}

\begin{tikzpicture}
\begin{axis}[
legend style={at={(1.05,1.00)},anchor=north west},
axis lines = left,
xlabel = \#threads,
ylabel = speedup,
xtick={0,1,2,3,4},
ytick={0,1,...,10},
]
\addplot [
color=blue,
mark=*,
dashed,
] plot coordinates {
	(0,0.0)
	(1,1.0)
	(2,1.0)
	(3,1.0)
	(4,1.0)
};
\addlegendentry{Without Accelerator Structure}
\addplot [
color=black,
mark=*,
dashed,
] plot coordinates {
	(0,0.0)
	(1,1.0)
	(2,1.0)
	(3,1.0)
	(4,1.0)
};
\addlegendentry{Regular Grid}
\addplot [
color=red,
mark=*,
dashed,
] plot coordinates {
	(0,0.0)
	(1,1.0)
	(2,1.0)
	(3,1.0)
	(4,1.0)
};
\addlegendentry{Octree}
\label{graph:NoShadows}
\end{axis}
\end{tikzpicture}

\subsection{Android Virtual Device in an HP Z440 desktop}

\begin{tikzpicture}
\begin{axis}[
legend style={at={(1.05,1.00)},anchor=north west},
axis lines = left,
xlabel = \#threads,
ylabel = speedup,
xtick={0,1,2,3,4},
ytick={0,1,...,10},
]
\addplot [
color=blue,
mark=*,
dashed,
] plot coordinates {
	(0,0.0)
	(1,1.0)
	(2,1.0)
	(3,1.0)
	(4,1.0)
};
\addlegendentry{Without Accelerator Structure}
\addplot [
color=black,
mark=*,
dashed,
] plot coordinates {
	(0,0.0)
	(1,1.0)
	(2,1.0)
	(3,1.0)
	(4,1.0)
};
\addlegendentry{Regular Grid}
\addplot [
color=red,
mark=*,
dashed,
] plot coordinates {
	(0,0.0)
	(1,1.0)
	(2,1.0)
	(3,1.0)
	(4,1.0)
};
\addlegendentry{Octree}
\label{graph:NoShadows}
\end{axis}
\end{tikzpicture}

\subsection{Android Virtual Device in a Clevo W230SS laptop}

\begin{tikzpicture}
\begin{axis}[
legend style={at={(1.05,1.00)},anchor=north west},
axis lines = left,
xlabel = \#threads,
ylabel = speedup,
xtick={0,1,2,3,4},
ytick={0,1,...,10},
]
\addplot [
color=blue,
mark=*,
dashed,
] plot coordinates {
	(0,0.0)
	(1,1.0)
	(2,1.0)
	(3,1.0)
	(4,1.0)
};
\addlegendentry{Without Accelerator Structure}
\addplot [
color=black,
mark=*,
dashed,
] plot coordinates {
	(0,0.0)
	(1,1.0)
	(2,1.0)
	(3,1.0)
	(4,1.0)
};
\addlegendentry{Regular Grid}
\addplot [
color=red,
mark=*,
dashed,
] plot coordinates {
	(0,0.0)
	(1,1.0)
	(2,1.0)
	(3,1.0)
	(4,1.0)
};
\addlegendentry{Octree}
\label{graph:NoShadows}
\end{axis}
\end{tikzpicture}

\section{Comparison with Android CPU Raytracer (\cite{Android_CPU_Raytracer})}

\par
Comparison ...